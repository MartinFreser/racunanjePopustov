\documentclass[12pt,a4paper,oneside,fleqn,openany]{book}
\usepackage[T1]{fontenc}
\usepackage[cp1250]{inputenc}
\usepackage{amsfonts}
\usepackage[slovene]{babel}

%\usepackage{algorithm2e}

\usepackage{longtable}
\renewcommand{\baselinestretch}{1}
\bibliographystyle{unsrt}
\usepackage{graphicx}


\usepackage{amsmath}
\usepackage{algorithmic}
\usepackage{algorithm}
\usepackage{verbatim}
\usepackage{tabularx}
\usepackage{color}
\usepackage{listings}
% \usepackage{breqn}

%
% Page setup
\setlength{\oddsidemargin}{0.5in}%
\setlength{\evensidemargin}{0.0in}%
\setlength{\topmargin}{0.0in}%
\setlength{\textwidth}{6.0in}%
\setlength{\textheight}{9.0in}%
%\setlength{\marginparsep}{3 mm}%
%\setlength{\marginparwidth}{1.5 cm}%
\raggedbottom
%
\begin{document}
\begin{large}PRILOGA K PROGRAMU: \textbf{DOKUMENTACIJA (model 1)}\end{large} \vspace{10mm}

\textbf{\textcolor{red}{Koraki programa:}}
\begin{enumerate}
	\item Izra�unamo $\sigma$ (standardni odklon) in $\mu$ (povpre�je) dolo�enega pretekla obdobja (obdobja na katerem se program u�i), katerega povpre�je �elimo zni�ati ali obdr�ati.
	\item Generiramo populacijo $P$ velikosti $n\geq 651$ (dolo�eno s pomo�jo tabele Cohen, 1988).
	\item Izra�unamo $k$ iz ena�be:
	$$ \mbox{povZp} = \frac{\mbox{realniVzorec}+k\cdot (\mbox{nakljucnaPopulacija}-\mbox{realniVzorec})}{n},$$
	kjer je povZp povpre�je iz to�ke 1, realniVzorec predstavlja popuste, ki so prispeli v teko�em oz. prej�njem dnevu, nakljucnaPopulacija=$P$ iz to�ke 2 in $n$ je velikost populacije iz to�ke 2.
	\item Za mejo $M$ postavimo vrednost $k\cdot(\mu+\sigma)$.
	\item Dobimo nove realne vrednosti, ki jih shranimo v realniVzorec.
	\item P=P-realniVzorec
	\item To�ke 3., 4., 5. in 6. ponavljamo dnevno.
\end{enumerate} \vspace{10mm}

Funkcija \emph{generirajModel}, ki je tudi glavna funkcija programa, ima naslednje parametre:
\begin{itemize}
	\item \emph{datumOd} - za�etni datum preteklega obdobja, na katerem se model "u�i"
	\item \emph{datumDo} - kon�ni datum preteklega obdobja, na katerem se model "u�i"
	\item \emph{datumOd2} - za�etni datum testiranja
	\item \emph{datumDo2} - kon�ni datum testiranja
	\item \emph{privzetaMejaZaAvtorizacijo} - sedanja oz trenutna fiksna privzeta meja za avtoriziranje
	\item \emph{dPovprecje} - za koliko odstotkov zelimo spremeniti povpre�je (�e ga ho�emo zmanj�ati, mora biti negativno, vrednosti morajo biti med 0 in 1)
	\item \emph{dSD} - analogno kot dPovpre�je, le da gre za standardni odklon
	\item \emph{dAvtorizacije} - Za koliko zni�amo popust, �e se znajde v avtorizaciji.
	\item \emph{mesto} oz. zavarovalni�ka enota (opcijsko)
\end{itemize} 

Funkcija \emph{dolociMejo} vrne mejo za avtoriziranje za izbran datum. Parametri te funkcije se prenesejo iz funkcije \emph{generirajModel}, tako da jih uporabnik ne rabi navajati.\\

Program vrne naslednje podatke:
\begin{itemize}
	\item stAvtorizacij (koliko avtorizacij je v testnem obdobju bilo potrebno) in stAvtorizacijPrivzeto (koliko avtorizacij bi bilo potrebno pri neki fiksni privzeti meji)
	\item Razliko med stAvtorizacij in stAvtorizacijPrivzeto in razmerje oz. odstotek zmanj�anja �tevila avtorizacij.
\end{itemize} \vspace{10mm}

\textbf{\textcolor{red}{Navodila uporabniku:}} \\

Uporabnik mora vnesti parametre funkcije \emph{generirajModel} (opis parametrov je zapisan pri opisu funkcije). Po vnosu parametrov program vrne \emph{mejo za avtoriziranje} za vsak dan testnega obdobja.

Primer preprostega klica funkcije:\\
\begin{lstlisting}[language=R]
simulator <- generirajModel("20140101","20140131",
	"20140201","20140228",	
	privzetaMejaZaAvtorizacijo = 0.37, dPovprecje = -0.02, 
	dSD = -0.03, dAvtorizacije = -0.05)
print(simulator)
\end{lstlisting}
Funkcija bo simulirala obdobje od 1. Februarja do 28. Februarja s podanimi parametri in nato shranila v objekt \texttt{simulator} �tevilo avtorizacij z uporabo meje, ter primerjavo z �tevilom avtorizacij, �e bi uporabljali privzeto mejo. Te podatke nato izpi�emo.\\

\end{document}
